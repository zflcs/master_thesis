% !TeX root = ../thuthesis-example.tex

% 中英文摘要和关键字

\begin{abstract}
  在请求完成后,I/O设备会发出一个中断信号来通知CPU,这会中断正在运行的任务并影响吞吐量,或者CPU主动轮询I/O设备的状态寄存器,消耗宝贵的CPU周期。在使用超高速I/O设备时,中断驱动和轮询机制的有害影响不容忽视。

  为了解决这个问题,我们将任务调度器的功能卸载到中断控制器(TAIC)。TAIC直接维护任务队列并唤醒被阻塞的任务,以实现快速唤醒机制。这种方法保护了CPU免受中断的影响,减少了CPU在中断响应以及维护调度队列方面的开销,从而降低了I/O响应延迟。我们将这一机制与网络卡驱动程序集成,并在FPGA平台上进行了验证。它减少了与中断相关的开销并降低了CPU利用率。在使用YCSB工作负载的Redis上的评估表明,TAIC实现了低尾部延迟,并将吞吐量提升了6\%。

  在通用操作系统中,随着并发量的增加,传统的多线程模型由于与内核多线程相关的高上下文切换成本而变得不足。在本文中,我们提出了一种新的并发模型,称为COPS。COPS采用基于优先级的协程模型作为基本任务单元,取代了高并发场景中的传统多线程模型,并为内核和用户空间协程提供了一个统一的基于优先级的调度框架。COPS将协程作为操作系统中的第一类公民引入,以提供异步I/O机制,使用内核协程作为I/O操作和设备之间的桥梁,以及用户协程来连接应用程序和操作系统服务。

  我们基于COPS设计了一个原型Web服务器,并在基于FPGA的系统上进行了广泛的实验,以评估COPS。结果表明,所提出的模型在保持相对较低的开销的同时,相比于多线程模型,在大型并发应用中实现了高达四倍的吞吐量。
  
  % 文章的总体思路:从应用场景对任务调度提出的需求,结合操作系统历史上对进程、线程等的定义,以及一些优秀论文中如何满足现实应用对任务调度提出的需求,引申出对任务模型的思考,随着硬件逐步发展,提出的操作系统、进程、线程、协程概念,对这些概念进行系统性的描述,描述软件的做法,现在有了硬件之后,区分清楚软件中定义的任务,硬件中定义的任务(硬件做的事情怎么与软件定义的任务进行结合?),各自的职责。对比出两者的区别后,从而给出一个软硬件的接口。根据这个抽象模型,在 QEMU 模拟器以及 FPGA 开发版上实现了对应的硬件,并完成了一些微基准测试,在 rel4 上进行了综合测试,得出了较为全面的性能对比(性能对比体现出特权级切换、上下文切换开销,中断延时等,从性能指标也体现出任务模型抽象),从而体现出优化的效果。

  % 关键词用“英文逗号”分隔,输出时会自动处理为正确的分隔符
  \thusetup{
    keywords = {关键词 1, 关键词 2, 关键词 3, 关键词 4, 关键词 5},
  }
\end{abstract}

\begin{abstract*}
  An abstract of a dissertation is a summary and extraction of research work and contributions.
  Included in an abstract should be description of research topic and research objective, brief introduction to methodology and research process, and summary of conclusion and contributions of the research.
  An abstract should be characterized by independence and clarity and carry identical information with the dissertation.
  It should be such that the general idea and major contributions of the dissertation are conveyed without reading the dissertation.

  An abstract should be concise and to the point.
  It is a misunderstanding to make an abstract an outline of the dissertation and words “the first chapter”, “the second chapter” and the like should be avoided in the abstract.

  Keywords are terms used in a dissertation for indexing, reflecting core information of the dissertation.
  An abstract may contain a maximum of 5 keywords, with semi-colons used in between to separate one another.

  % Use comma as separator when inputting
  \thusetup{
    keywords* = {keyword 1, keyword 2, keyword 3, keyword 4, keyword 5},
  }
\end{abstract*}
