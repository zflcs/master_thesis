\chapter{总结与展望}

本文在总结现有研究工作的基础上,提出了基于执行流的任务模型,

% 挑战
% 空间容量的挑战:目前,TAIC内部实现的队列正面临空间容量方面的挑战。随着队列长度的增加,所需的硬件资源也在增长。在实际应用场景中,可能需要处理成千上万的任务,这使得TAIC有限的内部空间容量难以满足如此庞大的请求数量。

% 为了解决这个问题,我们提出了一种新的设计,但尚未实施:在内存中维护就绪队列和阻塞槽,但确保它们只被TAIC访问,并向TAIC添加一个数据迁移组件。相应地调整TAIC的中断处理功能——数据迁移组件将阻塞任务的标识符从内存中的阻塞槽移动到内存中的就绪队列。这种方法旨在解决硬件空间容量问题,我们计划在后续工作中推进这一设计。

% 对延迟的影响:尽管TAIC能够处理中断并及时唤醒被阻塞的任务,但这并不能保证唤醒的任务可以立即执行。当当前CPU忙于处理其他具有更高优先级的任务时,这种情况可能会发生。如果有恶意任务(例如,可能由于紧密循环而不断占用CPU的任务),它们可能会独占CPU资源,导致系统延迟增加。在这些情况下,我们认为这些任务的行为是不适当的,因为它们没有与其他任务适当合作和共享资源。为了减少这种不良行为对系统延迟的影响,我们可以使用工具和机制来监控和限制任务的CPU使用。例如,编译器可以强制执行合作政策,确保任务在执行一定时间后将CPU让给其他任务(就像Concord[29]一样)。此外,运行时分析工具可以用来检测和识别长时间占用CPU的任务,以便进行调整或优化。此外,适当的优先级设置可以确保及时响应。

% 展望

% 使用场景:目前,我们仅在单核环境中使用TAIC。然而,TAIC在使用场景的选择上具有一定的灵活性。它不受CPU特权级别的限制,允许在用户空间使用。这使得用户进程能够通过将TAIC的地址空间映射到自己的空间中,构建高效的用户空间异步设备驱动程序。这种方法避免了传统内核旁路技术中检查设备状态的低效轮询方法,从而提高了性能。(与在用户空间使用uintr[2]相比,不同的路径通往同一个目标。)然而,当前的设计限制是TAIC在任何给定时刻只能被单个进程使用。当多个进程需要同时使用TAIC时,需要在内核中实现更严格的控制机制,以防止潜在的冲突和不可预测的行为。为了解决这个问题,我们计划在未来的研发中进一步完善设计,以支持多个进程安全高效地共享TAIC资源。

% 可扩展性:不仅外设,软件和定时器也可以是中断源。TAIC可以设计成允许软件通过写入相应的接口来触发软件中断。软中断在操作系统中扮演着重要角色,通常用于处理系统调用、进程间同步以及进程间通信(IPC)。如果使用TAIC来加速软中断的处理,不仅可以加快系统调用的响应速度,还可以提高IPC的效率。我们期望通过这些研究进一步提高系统的整体性能,特别是在高负载场景下,频繁的系统调用和IPC是必需的。此外,定时器中断作为操作系统调度机制的关键组成部分,直接影响操作系统的响应性和实时性能。因为内核调度器被卸载到TAIC中,所以在TAIC内处理定时器中断是可能的,这可以进一步优化操作系统的调度性能。探索TAIC在处理定时器中断中的应用将是我们未来研究的一个重要方向。我们期待通过这些研究更全面地增强系统的中断处理能力,从而为构建高效可靠的操作系统提供坚实的基础。

% TAIC,从内核中卸载了调度器的功能,是一个具有任务感知和任务调度能力的中断控制器。TAIC解决了CPU与I/O设备交互中的信息不对称问题。TAIC可以在中断处理例程中,不通过CPU从中断控制器获取信息,而是从其内部结构中获取中断相关信息,从而避免了CPU上下文切换的开销,并消除了CPU维护调度队列的需要。使用FPGA进行的评估表明,TAIC可以减少由中断引起的上下文切换开销,优化CPU利用率,并提高系统吞吐量。
